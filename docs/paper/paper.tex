\documentclass{article}
\usepackage{amsmath}
\usepackage{amsfonts}
\usepackage{graphicx}

\title{\textbf{ROUGH DRAFT!} \\ Term-Rewriting in the Letter String Domain}
\author{
  LSD Gang \\
  Department of Computer Science, Cal Poly Humboldt \\
  Arcata, CA, 95521 \\
  %% Email: bkovitz@humboldt.edu \\
}

\begin{document}
\maketitle
\begin{abstract}
    This paper presents a novel approach for constructing analogies in the letter string domain. We aim to enhance the understanding of semantic relationships between a strings of letters by leveraging term-rewriting. Our approach is evaluated against previous methodologies, demonstrating distinct advantages in accuracy and performance.
\end{abstract}

\tableofcontents

\pagebreak
\pagebreak

\section{Introduction}
The ability to construct analogies is fundamental to human cognitive processes...

\subsection{Previous models}
\begin{itemize}
    \item Douglas Hofstader / FARG
    \item Copycat, Metacat, ...
    \item LLMs
\end{itemize}

\subsection{How Painter-Canvas differs from previous models}

\section{Concepts}

\subsection{Analogy}
Why analogy is crucial for simulating human cognition.
Formal definition of an analogy.

\subsection{Chunk}
Formal definition of a chunk.

\subsection{Term-rewriting}
Formal definition of a chunk.


\section{Implementation}
The proposed algorithm was implemented using Python and tested on a dataset consisting of...
\\
(Pictures of PC-LSD GUI)

\section{Results}
In this section, we present the results obtained from our experiments. We compare our approach with existing methodologies on benchmark datasets.
\\
(Table of results)

\section{Discussion}
The results highlight...

\section{Conclusion}
This paper presents a novel approach for constructing analogies in the letter string domain...\\
\section{Future Research}
Future work will include... (ARC-AGI)

\section*{Acknowledgments}
We would like to thank...


\end{document}
